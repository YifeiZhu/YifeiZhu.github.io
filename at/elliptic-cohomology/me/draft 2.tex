\documentclass{gtpart}
\usepackage{amsmath,amssymb,amsthm,stmaryrd}
\usepackage[all]{xy}
\usepackage[usenames,dvipsnames]{xcolor}
\usepackage{tikz}
\usetikzlibrary{shapes,positioning,fit}
\usepackage{url}
\usepackage{hyperref}
\usepackage{enumerate}
\usepackage{tensor}
\usepackage{mathrsfs}
\usepackage{graphicx}
\usepackage{mathtools}
\usetikzlibrary{arrows}
%\usepackage{amsrefs}
%\usepackage{setspace}
%\doublespacing

\title{Modular equations for Lubin-Tate formal groups\\at chromatic level 2}
\author{Yifei Zhu}
\givenname{Yifei}
\surname{Zhu}
% \address{Department of Mathematics\\Northwestern University\\
%          Evanston\\IL 60208\\USA}
% \email{zyf@math.northwestern.edu}
%\urladdr{http://www.math.northwestern.edu/~zyf}

% \subject{primary}{msc2000}{55S25}
% \subject{secondary}{msc2000}{11F23, 11G18, 14L05, 55N20, 55N34, 55P43}

%\bibliographystyle{gtart}
\parskip 0.7pc
\parindent 0pt

\newtheorem{thm}[equation]{Theorem}
\newtheorem{cor}[equation]{Corollary}
\newtheorem{prop}[equation]{Proposition}
\newtheorem{lem}[equation]{Lemma}
\theoremstyle{definition}
\newtheorem{defn}[equation]{Definition}
\newtheorem{cstr}[equation]{Construction}
\theoremstyle{remark}
\newtheorem{rmk}[equation]{Remark}
\newtheorem{ex}[equation]{Example}
\newtheorem{case}[equation]{Case}
\newtheorem{slogan}[equation]{Slogan}
\newtheorem{ques}[equation]{Question}

\def\co{\colon\thinspace}
\newcommand{\mb}[1]{\mathbb{#1}}
\newcommand{\mf}[1]{\mathfrak{#1}}
\newcommand{\Hom}{\ensuremath{{\rm Hom}}}
\newcommand{\Aut}{{\rm Aut}}
\newcommand{\Gal}{{\rm Gal}}
\newcommand{\LT}{{\rm LT}}
\newcommand{\Spec}{{\rm Spec\thinspace}}
\newcommand{\Proj}{{\rm Proj\thinspace}}
\newcommand{\Spf}{{\rm Spf\thinspace}}
\newcommand{\Ell}{{\rm Ell}}
\newcommand{\Sch}{{\rm Sch}}
\newcommand{\cF}{\overline {\mb F}}
\newcommand{\ck}{\overline k}
\newcommand{\CA}{{\cal A}}
\newcommand{\CB}{{\cal B}}
\newcommand{\CC}{{\cal C}}
\newcommand{\CE}{{\cal E}}
\newcommand{\CF}{{\cal F}}
\newcommand{\CG}{{\cal G}}
\newcommand{\CH}{{\cal H}}
\newcommand{\CHom}{{\cal H}om}
\newcommand{\CLie}{{\cal L}ie}
\newcommand{\CM}{{\cal M}}
\newcommand{\CMB}{\overline {\cal M}}
\newcommand{\CO}{{\cal O}}
\newcommand{\CP}{{\cal P}}
\newcommand{\CS}{{\cal S}}
\newcommand{\FF}{{\mf F}}
\newcommand{\Mod}{{\rm Mod}}
\newcommand{\Alg}{{\rm Alg}}
\newcommand{\dl}{{\rm DL}}
\newcommand{\Set}{{\rm Set}}
\newcommand{\Sq}{{\rm Sq}}
\newcommand{\Frob}{{\rm Frob}}
\newcommand{\DF}{{{\rm DefFrob}_\BG}}
\newcommand{\Model}{{\rm Model}}
\newcommand{\HGa}{{\widehat{\mb G}_a}}
\newcommand{\HGm}{{\widehat{\mb G}_m}}
\newcommand{\Gm}{{{\mb G}_m}}
\newcommand{\DL}{Dyer-Lashof~}
\newcommand{\EM}{Eilenberg-Mac~Lane~}
\newcommand{\BC}{{\mb C}}
\newcommand{\BE}{{\mb E}}
\newcommand{\BF}{{\mb F}}
\newcommand{\BG}{{\mb G}}
\newcommand{\BN}{{\mb N}}
\newcommand{\BP}{{\mb P}}
\newcommand{\BQ}{{\mb Q}}
\newcommand{\BR}{{\mb R}}
\newcommand{\BS}{{\mb S}}
\newcommand{\BW}{{\mb W}}
\newcommand{\BZ}{{\mb Z}}
\newcommand{\fm}{{\mf m}}
\newcommand{\HC}{\widehat{C~}\!}
\newcommand{\HCC}{\widehat{~\!\!\CC}}
\newcommand{\HE}{\widehat{E~}\!}
\newcommand{\Hf}{\widehat{f}}
\newcommand{\Hphi}{\widehat{\phi}}
\newcommand{\Hpsi}{\widehat{\psi}}
\newcommand{\HS}{\widehat{S~}\!}
\newcommand{\TA}{\tilde{\A}}
\newcommand{\Tc}{\tilde{c}}
\newcommand{\TE}{\widetilde{E\thinspace}\!}
\newcommand{\Tf}{\widetilde{f}}
\newcommand{\Tp}{\widetilde{\psi}}
\newcommand{\TW}{\widetilde{W\thinspace}\!}
\newcommand{\tA}{\widetilde{\A}}
\renewcommand{\th}{\widetilde{h}}
\newcommand{\tj}{\widetilde{j}}
\newcommand{\tw}{\widetilde{w}}
\newcommand{\md}{~~{\rm mod}~}
\newcommand{\ad}{\text{and}}
\newcommand{\can}{{\rm can}}
\newcommand{\HT}{{\rm ht}}
\newcommand{\id}{{\rm id}}
\newcommand{\op}{{\rm op}}
\newcommand{\tf}{{\rm tf}}
\newcommand{\TMF}{{\rm TMF}}
\newcommand{\tmf}{{\rm tmf}}
\newcommand{\MF}{{\rm MF}}
\newcommand{\tr}{{\rm trace}}
\newcommand{\univ}{{\rm univ}}
\newcommand{\Ext}{{\rm Ext}}
\newcommand{\Tor}{{\rm Tor}}
\newcommand{\nul}{{\rm nul}}
\newcommand{\Sub}{{\rm Sub}}
\newcommand{\A}{\alpha}
\newcommand{\B}{\beta}
\renewcommand{\D}{\Delta}
\renewcommand{\d}{\delta}
\newcommand{\f}{\phi}
\newcommand{\G}{\Gamma}
\newcommand{\g}{\gamma}
\newcommand{\K}{\kappa}
\renewcommand{\l}{\lambda}
\renewcommand{\o}{\omega}
\newcommand{\ou}{\underline{\omega\!}}
\newcommand{\si}{\sigma}
\newcommand{\T}{\tau}
\newcommand{\om}{\underline{\omega\!}_{~E/S}}
\newcommand{\p}{\psi^3}
\newcommand{\s}{S^\bullet}
\newcommand{\ce}{\coloneqq}
\newcommand{\lb}{\llbracket}
\newcommand{\rb}{\rrbracket}
\newcommand{\lp}{(\!(}
\newcommand{\rp}{)\!)}
\newcommand{\Ht}{\widehat{T}}
\newcommand{\Tt}{\widetilde{T}}
\newcommand{\mt}{\widetilde{m}}
\newcommand{\lt}{\widetilde{\lambda}}
\newcommand{\todo}{\spadesuit}
\newcommand{\totodo}{\heartsuit}
\renewcommand{\=}{\approx}
\renewcommand{\-}{\sim}
\newcommand{\isog}[1]{Proposition \ref{prop:isog}\thinspace \eqref{isog(#1)}}
\newcommand{\q}[1]{Proposition \ref{prop:Q}\thinspace \eqref{Q(#1)}}
\newcommand{\go}[1]{Definition \ref{def:go}\thinspace \eqref{go(#1)}}
\newcommand{\rd}[1]{\textcolor{red}{#1}}
\newcommand{\bl}[1]{\textcolor{blue}{#1}}
\newcommand{\wt}[1]{\textcolor{white}{#1} \!~}
\newcommand{\gl}{{\rm gl}}
\newcommand{\GL}{{\rm GL}}
\newcommand{\SL}{{\rm SL}}
\newcommand{\Tate}{{\rm Tate}}
\newcommand{\ch}[2]{{#1 \choose #2}}

\makeatletter
\DeclareRobustCommand\widecheck[1]{{\mathpalette\@widecheck{#1}}}
\def\@widecheck#1#2{%
    \setbox\z@\hbox{\m@th$#1#2$}%
    \setbox\tw@\hbox{\m@th$#1%
       \widehat{%
          \vrule\@width\z@\@height\ht\z@
          \vrule\@height\z@\@width\wd\z@}$}%
    \dp\tw@-\ht\z@
    \@tempdima\ht\z@ \advance\@tempdima2\ht\tw@ \divide\@tempdima\thr@@
    \setbox\tw@\hbox{%
       \raise\@tempdima\hbox{\scalebox{1}[-1]{\lower\@tempdima\box
\tw@}}}%
    {\ooalign{\box\tw@ \cr \box\z@}}}
\makeatother

\numberwithin{equation}{section}
\renewcommand{\theequation}{\thesection.\arabic{equation}}


\begin{document}



\begin{abstract}
 We give an integral lift of the Kronecker congruence for moduli of finite 
 subgroups of elliptic curves.  This leads to a uniform presentation for the 
 power operation structure on Morava $E$-theories of height 2.  
\end{abstract}

\maketitle



\section{Introduction}

\subsection{Moduli of elliptic curves and of formal groups}

The Kronecker congruence 
\begin{equation}
 \label{Kronecker}
 \big(\,\tj - \,j^{\,p}\big) \,\! \big(\,\tj^{\,p} - \,j\,\big) \equiv 0 \md p 
\end{equation}
gives an equation for a curve that represents the moduli problem $[\G_0(p)]$ for 
elliptic curves over a perfect field of characteristic $p$.  This moduli problem 
associates to such an elliptic curve its finite flat subgroup schemes of rank 
$p$.  A choice of such a subgroup scheme is equivalent to an isogeny from the 
elliptic curve with a prescribed kernel.  The $j$-invariants of the source and 
target curves of this isogeny are parametrized by the coordinates $j$ and $\tj$.  

More precisely, the Kronecker congruence provides a {\em local} description for 
$[\G_0(p)]$ at a supersingular point.  For large primes $p$, the supersingular 
locus at $p$ may consist of more than one closed point.  In this case, the 
modular curve does not have an equation in the simple form above; only its 
completion at a single supersingular point does.  

Integrally, there are polynomials that describe the modular curve for 
$[\G_0(p)]$ over $\Spec \BZ$.  The ``classical modular polynomials'' lift and 
globalize the Kronecker congruence, while the ``canonical modular polynomials,'' 
in a different pair of parameters, appear simpler (see the Modular Polynomial 
Databases in \href{http://magma.maths.usyd.edu.au/magma/handbook/modular_curves}
{Magma} for the terminology and numerical examples).  Computing these modular 
equations for $[\G_0(p)]$ can be difficult.  As Milne warns in 
\cite[Section 6]{Milne}, ``one gets nowhere with brute force methods in this 
subject.''  

On a related subject, Lubin and Tate developed the deformation theory for 
one-dimensional formal groups of finite height \cite{LubinTate}.  Later, with 
motivation from algebraic topology, Strickland studied the classification of 
finite subgroups of Lubin-Tate universal deformations.  In particular, he proved 
a representability theorem for this moduli of deformations 
\cite[Theorem 42]{Str97}.  

At height 2, there is a connection between the moduli of formal groups and the 
moduli of elliptic curves.  This is the Serre-Tate theorem, which states that 
$p$-adically, the deformation theory of an elliptic curve is equivalent to the 
deformation theory of its $p$-divisible group \cite[Section 6]{LST}.  In 
particular, the $p$-divisible group of a supersingular elliptic curve is formal.  

Thus the local information provided by the Kronecker congruence (and its 
integral lifts) becomes useful for understanding deformations of formal groups 
of height 2.  In this paper, we give an explicitation for the representing 
formal scheme in Strickland's theorem.  Equivalently, this describes the 
complete local ring of $[\G_0(p)]$ at a supersingular point, i.e., relative to 
the universal formal deformation of a supersingular elliptic curve in 
characteristic $p$, as studied by Katz and Mazur \cite{KM}.  

\begin{thm}
 \label{thm:me}
 Let $\BG_0$ be a formal group over $\cF_p$ of height $2$, and let $\BG$ be its 
 universal deformation.  Write $A_m$ for the ring $\CO_{\Sub_m(\BG)}$ studied in 
 \cite{Str97}, which classifies degree-$p^m$ subgroups of the formal group 
 $\BG$.  In particular, write $A_0 \cong \BW\big(\cF_p\big)\lb h \rb$ according 
 to the Lubin-Tate theorem.  

 Then the ring $A_1 \cong \BW\big(\cF_p\big)\lb h, \A \rb / \big(w(h,\A)\big)$ 
 is determined by the polynomial 
 \begin{equation}
  \label{me}
  w(h,\A) = (\A - p) \big(\A + (-1)^{\,p}\big)^p 
            - \big(h - p^2 + (-1)^{\,p}\big) \A 
 \end{equation}
 which reduces to $\A (\A^{\,p} - h)$ modulo $p$.  
\end{thm}

\begin{rmk}
 As a result of a different choice of parameters, the last congruence above is 
 not in the form of Kronecker's (cf.~\cite[Remark 7.7.1]{KM} and see Section 
 \ref{sec:parameter} below).  In fact, let $\tA$ denote the image of $\A$ under 
 the Atkin-Lehner involution, so that $\A \cdot \tA = (-1)^{\,p + 1} p$, which 
 is the constant term of \eqref{me} as a polynomial in $\A$.  Dividing a factor 
 of $\A$ from the modular equation $w(h,\A) = 0$, we obtain a congruence 
 \[
  h \equiv \A^{\,p} + \tA \md p 
 \]
 This is a manifest of the Eichler-Shimura relation $T_p \equiv F + V \md p$ 
 between the Hecke, Frobenius, and Verschiebung operators, which reinterprets 
 $[\G_0(p)]$ in characteristic $p$.  

 The polynomial $w(h,\A)$ can be viewed as a local variant of a canonical 
 modular polynomial, whose parameters are the $j$-invariant and a certain 
 eta-quotient.  See \cite[Example 2.4, esp.~(2.4)]{Choi} for an algorithm that 
 we will adapt to prove Theorem \ref{thm:me} in Section \ref{sec:pf} below.  
 Related to this, compare the sequences $\{\,j_n^{\,(p)}(z)\}_{n = 1}^\infty$ in 
 \cite{Ahlgren}, with $p \in \{2,3,5,7,13\}$, and $\{\,j_m(z)\}_{m = 0}^\infty$ 
 in \cite{BKO} of Hecke translates of Hauptmoduln, which explain the relation 
 $h = T_p \, \A$ from above.  
\end{rmk}



\subsection{Algebras of cohomology operations}

The Adem relations 
\begin{equation}
 \label{Adem}
 \Sq^i\,\Sq^{\,j} = \sum_{k = 0}^{\left[\frac{i}{2}\right]} 
                    \ch{j - k - 1}{i - 2 k} \Sq^{i + j - k}\,\Sq^k 
 \hskip 2cm 0 < i < 2\,j 
\end{equation}
describe the multiplication rule for the Steenrod squares $\Sq^i$.  These are 
power operations in ordinary cohomology with $\BZ/2$-coefficients.  In general, 
for ordinary cohomology with $\BZ/p$-coefficients, the collection of its power 
operations has the structure of a Steenrod algebra.  

Quillen's work connects complex cobordism and the theory of one-dimensional 
formal groups \cite{Quillen}.  From this viewpoint, ordinary cohomology theories 
with $\BZ/p$-coefficients fit into a framework of chromatic homotopy theory, as 
theories concentrated at height $\infty$.  

The power operation algebras for cohomology theories at other chromatic levels 
have been studied as well.  In particular, key to the chromatic viewpoint is a 
family of Morava $E$-theories, one for each finite height $n$ at a particular 
prime $p$.  More precisely, given any formal group $\BG_0$ of height $n$ over a 
perfect field of characteristic $p$, there is a Morava $E$-theory associated to 
the Lubin-Tate universal deformation of $\BG_0$.  Via Bousfield localizations, 
these Morava $E$-theories determine the chromatic filtration of the stable 
homotopy category.  

There is a connection between (stable) power operations in a Morava $E$-theory 
$E$ and deformations of powers of Frobenius on its corresponding formal group 
$\BG_0$.  This is Rezk's theorem, built on the work of Ando, Hopkins, and 
Strickland \cite{Ando95, Str97, Str98, AHS04}.  It gives an equivalence of 
categories between (i) graded commutative algebras over a \DL algebra for $E$ 
and (ii) quasicoherent sheaves of graded commutative algebras over the moduli 
problem of deformations of $\BG_0$ and Frobenius isogenies 
\cite[Theorem B]{cong}.  Here, the \DL algebra is a collection of power 
operations that governs all homotopy operations on commutative $E$-algebra 
spectra \cite[Theorem A]{cong}.  

At height 2, information from the moduli of elliptic curves allows a concrete 
understanding of the power operation structure on Morava $E$-theories.  Rezk 
computed the first example of a presentation for an $E$-\DL algebra, in terms of 
explicit generators and quadratic relations analogous to the Adem relations 
\eqref{Adem} \cite{h2p2}.  Moreover, he gave a uniform presentation, which 
applies to $E$-theories at all primes $p$, for the mod-$p$ reduction of their 
\DL algebras \cite[4.8]{mc1}.  Underlying this presentation is the Kronecker 
congruence \eqref{Kronecker} \cite[Proposition 3.15]{mc1}.  

In this paper, we provide an ``integral lift'' of Rezk's presentation, in the 
same sense that Theorem \ref{thm:me} above lifts the Kronecker congruence.  We 
start with the following.  

\begin{thm}
 \label{thm:po}
 Let $E$ be a Morava $E$-theory spectrum of height $2$ at the prime $p$.  There 
 is a total power operation 
 \[
  \begin{split}
                \psi^p \co E^0 \to & ~ E^0(B\Sigma_p) / I \\
   \BW\big(\cF_p\big)\lb h \rb \to & ~ \BW\big(\cF_p\big)\lb h,\A \rb 
                                       / \big( w(h,\A) \big) 
  \end{split}
 \]
 where $I$ is an ideal of transfers.  

 \begin{enumerate}[{\em (i)}]
  \item The polynomial 
  \[
   w(h,\A) = w_{p + 1} \A^{\,p + 1} + \cdots + w_1 \A + w_0 
   \hskip 2cm w_i \in E^0 
  \]
  can be given as \eqref{me} from Theorem \ref{thm:me}.  In particular, 
  $w_{p + 1} = 1$, $w_1 = -h$, $w_0 = (-1)^{\,p + 1} p$, and the remaining 
  coefficients 
  \[
   w_i = (-1)^{\,p\,(\,p - i + 1)} \left[ \ch{p}{i - 1} 
         + (-1)^{\,p + 1} \, p \, \ch{\,p\,}{i} \right] 
  \]

  \item The image $\psi^p(h) = \sum_{i = 0}^p Q_i(h) \, \A^i$ is then given by 
  \[
   \psi^p(h) = \A + \sum_{i = 0}^p \A^i \sum_{\T = 1}^p w_{\T + 1} \, d_{i,\T} 
  \]
  where 
  \[
   d_{i,\T} = \sum_{n = 0}^{\T - 1} (-1)^{\T - n} \, w_0^n 
              \sum_{\stackrel{\scriptstyle m_1 + \cdots + m_{\T - n} = \T + i} 
              {1 \,\leq\, m_s \,\leq\, m_{s + 1} \,\leq\, p + 1}} w_{m_1} \cdots 
              w_{m_{\T - n}} 
  \]
  In particular, $Q_0(h) \equiv h^{\,p} \md p$.  
 \end{enumerate}
\end{thm}

This leads to the following.  

\begin{thm}
 \label{thm:DL}
 Continue with the notation in Theorem \ref{thm:po}.  Let $\G$ be the \DL 
 algebra for $E$, in the sense of \cite{cong}, which is the ring of additive 
 power operations on $K(2)$-local commutative $E$-algebras.  

 Then $\G$ admits a presentation as the associative ring generated over 
 $E^0 \cong \BW\big(\cF_p\big)\lb h \rb$ by elements $Q_i$, $0 \leq i \leq p$, 
 subject to a set of relations.  

 \begin{enumerate}[{\em (i)}]
  \item Adem relations 
  \[
   Q_k Q_0 = -\sum_{j = 1}^{p - k} w_0^{\,j} \, Q_{k + j} Q_j \, 
             - \sum_{j = 1}^p \sum_{i = 0}^{j - 1} w_0^i \, d_{k,\,j - i} \, 
             Q_i Q_j \hskip 2.3cm \text{for $1 \leq k \leq p$} 
  \]
  where the first summation is vacuous if $k = p$.  

  \item Commutation relations 
  \[
   \begin{split}
    Q_i \, c = & ~ (F c) \, Q_i ~~~ \text{for $\, c \in \BW\big(\cF_p\big)$ and 
                 all $i$, with $F$ the Frobenius automorphism} \\
    Q_0 \, h = & ~ e_0 + (-1)^{\,p + 1} r \sum_{m = 0}^{p - 1} s^m e_{p + m + 1} 
                 + (-1)^{\,p} \Bigg( e_p + r \, e_{2 p} 
                 + \sum_{m = 1}^p s^m e_{p + m} \Bigg) \\
               & + \sum_{j = 1}^{p - 1} \, (-1)^{\,p \, j} \Bigg[ e_j 
                 + r \, s^{\,p - j} e_{2 p} + r \! \sum_{m = 0}^{p - j - 1} s^m 
                   \big(e_{p + j + m} + (-1)^{\,p + 1} e_{p + j + m + 1}\big) 
                   \Bigg] \\
    Q_k \, h = & ~ (-1)^{\,p\,(p - k)} \ch{p}{k} \Bigg( e_p + r \, e_{2 p} 
                 + \sum_{m = 1}^p s^m e_{p + m} \Bigg) 
                 + \sum_{j = k}^{p - 1} \, (-1)^{\,p\,(\,j - k)} \ch{\,j\,}{k} 
                   \Bigg[ e_j \\
               & + r \, s^{\,p - j} e_{2 p} + r \! \sum_{m = 0}^{p - j - 1} s^m 
                   \big(e_{p + j + m} + (-1)^{\,p + 1} e_{p + j + m + 1}\big) 
                   \Bigg] ~~~\, \text{for \,\! $0 < k < p$} \\
    Q_p \, h = & ~ e_p + r \, e_{2 p} + \sum_{m = 1}^p s^m e_{p + m} 
   \end{split}
  \]
  where $r = h - p^2 + (-1)^{\,p}$, $s = p + (-1)^{\,p}$, and 
  \[
   \begin{split}
    e_n = & ~ \sum_{m = n}^{p + 1} (-1)^{(p + 1) (m - n)} \ch{m}{n} Q_{m - 1} \\
          & + \sum_{m = n}^{2 p} (-1)^{(p + 1) (m - n)} \ch{m}{n} 
              \sum_{\stackrel{\scriptstyle i + j = m} 
              {0 \,\leq\, i, \, j \,\leq\, p}} \sum_{\T = 1}^p 
              w_{\T + 1} \, d_{i,\T} \, Q_j 
   \end{split}
  \]
  the first summation for $e_n$ being vacuous if $p + 2 \leq n \leq 2\,p$, and 
  being vacuous in its term $m = 0$ if $n = 0$.  
 \end{enumerate}
\end{thm}

\begin{rmk}
 The \DL algebra $\G$ has the structure of a twisted bialgebra over $E^0$.  The 
 ``twists'' are described by the commutation relations above.  The product 
 structure is encoded in the Adem relations.  Certain Cartan formulas give rise 
 to the coproduct structure.  

 In Section \ref{subsec:comm} below, we will present a proof for the commutation 
 relations, in such a way that the same method applies to give the Cartan 
 formulas: for each $0 \leq k \leq p$, $Q_k(x y)$ equals the expression on the 
 right-hand side of the commutation relation for $Q_k \, h$, where 
 $r = h - p^2 + (-1)^{\,p}$ and $s = p + (-1)^{\,p}$ as above, and 
 \[
  e_n = \sum_{m = n}^{2 p} (-1)^{(p + 1) (m - n)} \ch{m}{n} 
  \sum_{\stackrel{\scriptstyle i + j = m}{0 \,\leq\, i, \, j \,\leq\, p}} 
  Q_i(x) \, Q_j(y) 
 \]
\end{rmk}



\subsection{Acknowledgements}

I thank Paul Goerss, Charles Rezk, and Joel Specter for helpful discussions.  

I thank my friends Tzu-Yu Liu and Meng Yu for their continued support, 
especially during me sketching Section \ref{sec:pf} in their home in California.  



\section{Parameters in a modular equation}
\label{sec:parameter}

In this section, we discuss preliminaries needed for proving the main results.  
For some of the details, with examples, we refer the reader to 
\cite[Sections 2 and 3]{ho}.  

Given a formal group $\BG_0$ over $\cF_p$ of height 2, let $\BG$ be its 
universal deformation over the Lubin-Tate ring $\BW\big(\cF_p\big) \lb h \rb$.  
Let $E$ be a Morava $E$-theory spectrum of height 2 at the prime $p$, such that 
$E^0 \cong \BW\big(\cF_p\big) \lb h \rb$ and $\Spf E^0 \BC\BP^\infty \cong \BG$.  

Recall from \cite[Section 2.1]{ho} that there are {\em $\CP_N$-models} for the 
above data, constructed from the moduli of smooth elliptic curves, where $N > 3$ 
are integers prime to $p$.  Specifically, each $\CP_N$ is a moduli problem that 
encodes the choice of a point of exact order $N$ and a nonvanishing one-form.  
It is represented by a scheme $\CM_N$ over $\BZ[1/N]$.  Via the Serre-Tate 
theorem, the formal group of the representing elliptic curve is isomorphic to 
the universal deformation $\BG$ of $\BG_0$.  Up to isomorphism, the $E$-theory 
is independent of the choice of a $\CP_N$-model.  

The purpose of choosing a $\CP_N$-model is to enable explicit calculations for 
power operations in the $E$-theory.  In particular, there are parameters from 
the moduli of elliptic curves that are both geometrically interesting and 
computationally convenient.  

Recall from \cite[Section 3.1]{ho} that a total power operation $\psi^p$ on 
$E^0$ corresponds to a deformation of $p$-power Frobenius $\Psi_N^{(p)}$ on the 
universal $\CP_N$-curve \cite[Theorem B]{cong}.  In particular, under this 
correspondence, we choose a pair of parameters $h$ (as for $E^0$ above) and $\A$ 
for $\psi^p$ as follows, which translate respectively to the {\em deformation} 
parameter $\rm T$ and the {\em norm} parameter $\rm {\bf N}(X(P))$ for 
$[\G_0(p)]$ at a supersingular point in \cite[Section 7.7]{KM}.  

The simultaneous moduli problem $\big(\CP_N,[\G_0(p)]\big)$ is represented by a 
scheme $\CM_{N,\,p}$, which is finite and flat over $\CM_N$ of degree $p + 1$.  
Upon formal completion at the supersingular point corresponding to $\BG_0$, the 
scheme $\CM_{N,\,p}$ is isomorphic to the formal spectrum of the target ring of 
$\psi^p$ \cite[Theorem 1.1]{Str98}.  

Via a {\em dehomogenization} procedure, the parameters $h$ and $\A$ are 
constructed from weakly holomorphic modular forms $H$ (a factor of a Hasse 
invariant) and $\K$ on $\G_1(N) \times \G_0(p)$ (see 
\cite[Proposition 2.8 and Examples 2.6 and 3.4]{ho}).  Moreover, the 
{\em Atkin-Lehner involution} of $[\G_0(p)]$ gives a new pair of parameters 
$\th$ and $\tA$ for $\psi^p$ (cf.~\cite[11.3.1]{KM}).  In fact, they are the 
images of $h$ and $\A$ under $\psi^p$.  

As a result of Lubin's isogeny construction in 
\cite[proof of Theorem 1.4]{Lubin}, the parameter $\A$ plays a double role, 
which is important for our application.  Algebraically, it is a norm by 
construction and is hence $\G_0(p)$-invariant \cite[Construction 3.1\,(ii)]{ho}.  
Geometrically, it defines the cotangent map to the isogeny $\Psi_N^{(p)}$ 
\cite[Remark 3.2]{ho}.  In particular, this leads to the following.  

\begin{prop}
 \label{prop:cusps}
 Let $\CM_N$ and $\CM_{N,\,p}$ be the moduli schemes defined above.  In a 
 punctured formal neighborhood of the cusps $\CMB_N - \CM_N$, the scheme 
 $\CM_{N,\,p}$, viewed as a relative curve over $\CM_N$, has an equation 
 \[
  (\A - p) \big(\A + (-1)^{\,p}\big)^p = 0 
 \]
\end{prop}

\begin{proof}
 Choose the particular local coordinate in \cite[Theorem 4]{Ando95} on the 
 universal $\CP_N$-curve.  Let $\A$ be the norm parameter constructed from this 
 coordinate.  In particular, $\A \cdot \tA = (-1)^{\,p + 1} p$.  

 In view of the geometric interpretation for $\A$ above and the transformation 
 of bases for cotangent spaces in \cite[Remark 3.16]{ho}, we see that the stated 
 equation follows from the discussion in the first new paragraph on page Ka-23 
 of \cite{padicprop}---there the $(\,p + 1)$ roots of this equation are given 
 (in Katz's notation, $\ell \ce p$ and $n \ce N$).  Note that when $p = 2$, as a 
 result of choice of local coordinates, the isogeny $\pi$ in 
 \cite[Section 1.11]{padicprop} differs by a sign from the restriction of the 
 isogeny $\Psi_N^{(p)}$ around the ramified cusp of $\G_0(p)$.  
\end{proof}



\section{Proof of Theorem \ref{thm:me}}
\label{sec:pf}

Choose a $\CP_N$-model for $\BG$.  In the scheme $\CM_{N,\,p}$ representing the 
simultaneous moduli problem $\big(\CP_N,[\G_0(p)]\big)$, there exists a 
supersingular point in characteristic $p$ whose corresponding $j$-invariant lies 
in $\BF_p \subset \BF_{p^2}$ (see, e.g., 
\cite[Theorem 14.18 and Proposition 14.15]{Cox}).  Let $j_0 \in \BZ$ be a lift 
of this $j$-invariant.  Consider a formal neighborhood $U$ that contains this 
single supersingular point.  Note that $U \cong \Spf A_1$ and that it is 
preserved under the Atkin-Lehner involution.  

Define a modular function $h \ce j - j_0$, where $j\,(z) = q^{-1} + 744 + O(q)$ 
as usual.  It then serves as a deformation parameter for $A_0$ and $A_1$.  Let 
$\A$ be a norm parameter for $A_1$.  Thus there exists a unique polynomial 
\begin{equation}
 \label{w}
 w(h,\A) = \A^{\,p + 1} + \sum_{i = 0}^p w_i \, \A^i 
\end{equation}
with $w_i \in \BW\big(\cF_p\big)\lb h \rb$ such that 
$A_1 \cong A_0[\A] / \big(w(h,\A)\big)$.  Write $\th$ and $\tA$ for the images 
of $h$ and $\A$ under the Atkin-Lehner involution.  

Given the geometric interpretation of $\tA$ in Section \ref{sec:parameter} and 
in view of the Hasse invariant as defined in \cite[12.4.1]{KM}, since the 
deformation parameter $h = j - j_0$, the modular function $\tA$ on $\G_0(p)$ has 
a $q$-expansion 
\[
 \tA(z) = \mu (q^{-1} + a_0) + O(q) = \mu \, q^{-1} + O(1) 
\]
for some unit $\mu \in \BW\big(\cF_p\big)^{\!\times} \!\cap \BZ$ and 
$a_0 \in \BZ$ such that $a_0 \equiv 744 - j_0 \md p$.  

Thus for $2 \leq i \leq p$, there exist constants $\tw_i \in p\BZ$ such that 
\[
 \tA^{\,p} + \tw_p \, \tA^{\,p - 1} + \cdots + \tw_2 \, \tA = \mu^{\,p} q^{-p} 
 + O(1) 
\]
On the other hand, we have 
\[
 \th\,(z) = j\,(p z) - j_0 = q^{-p} + O(1) 
\]
Comparing the two displays above, we then have 
\[
 \tA^{\,p} + \tw_p \, \tA^{\,p - 1} + \cdots + \tw_2 \, \tA = \mu^{\,p} \, \th 
 + K + O(q) 
\]
for some $K \in \BZ$.  Passing to the mod-$p$ reduction of this identity, we see 
that $K \in p\BZ$.  

Therefore, by an abuse of notation, we can instead choose a deformation 
parameter $h$ (and $w_i$ in \eqref{w} accordingly) such that 
\[
 \tA^{\,p} + \tw_p \, \tA^{\,p - 1} + \cdots + \tw_2 \, \tA = \th + O(q) 
\]
without changing the expressions for $A_0$ and $A_1$ (note that $\A$ and $\tA$ 
are both independent of the choice of $h$).  From this we obtain 
\[
 \tA^{\,p + 1} + \tw_p \, \tA^{\,p} + \cdots + \tw_2 \, \tA^2 = \th \, \tA 
 + O(1) 
\]
In view of the expression (under the Atkin-Lehner involution) for $A_1$ as a 
free module over $A_0$ of rank $p + 1$, and in view of the $q$-expansions for 
$\th$ and $\tA$, we see that the last term $O(1)$ above must be constant.  

Applying the Atkin-Lehner involution to this polynomial relation between $\th$ 
and $\tA$, we then conclude that except for $i = 1$, the coefficients $w_i$ in 
\eqref{w} are all constants, or more precisely, $w_i \in \BZ$.  It remains to 
determine their values, which follows from Proposition \ref{prop:cusps} by 
continuity of modular functions over the moduli scheme.  \qed



\section{Proof of Theorems \ref{thm:po} and \ref{thm:DL}}

Theorem \ref{thm:po}\,(i) follows from \cite[Theorem 1.1]{Str98}.  We show the 
remaining parts in two steps.  

\subsection{The total power operation formula and the Adem relations}

To compute $\psi^p(h)$, we follow the recipe illustrated in 
\cite[Example 3.4]{ho} and generalize \cite[proof of Proposition 6.4]{ho}.  
Since 
\[
 \begin{split}
  w(h,\A) = & ~ w_{p + 1} \A^{\,p + 1} + \cdots + w_1 \A + w_0 \\
          = & ~ w_{p + 1} \A^{\,p + 1} + \cdots - h \, \A + \tA \, \A 
 \end{split}
\]
is zero in the target ring of $\psi^p$, we have 
\[
 \hskip .5cm h = w_{p + 1} \A^{\,p} + \cdots + w_2 \, \A + \tA 
\]
where $w_{p + 1}$, \ldots, $w_2$ are constants, i.e., they do not involve $h$, 
as computed in Theorem \ref{thm:po}\,(i).  Applying the Atkin-Lehner involution 
to this identity, we then get 
\[
 \hskip -.8cm \psi^p(h) = \th = w_{p + 1} \, \tA^{\,p} + \cdots + w_2 \, \tA 
 + \A 
\]
For $1 \leq \T \leq p$, we need only express each $\tA^\T$ as a polynomial in 
$\A$ of degree at most $p$ with coefficients in $E^0$.  The constant terms 
$d_{0,\T}$ of these polynomials have been computed as $d_\T$ in 
\cite[proof of Proposition 6.4]{ho}.  The same method there applies to give the 
stated formulas for the higher coefficients $d_{i,\T}$ with $1 \leq i \leq p$.  

To derive the Adem relations, we generalize 
\cite[proof of Proposition 3.6\,(iv)]{p3} 
(cf.~\cite[proof of Proposition 6.4]{ho}).  
In view of the relation $\A \cdot \tA = w_0$, we have 
\[
 \begin{split}
  \psi^p\big(\psi^p(x)\big) = & ~ \psi^p \sum_{j = 0}^p Q_j(x) \, \A^{\,j} \\
                            = & ~ \sum_{j = 0}^p \psi^p\big(Q_j(x)\big) \, 
                                  \psi^p(\A)^{\,j} \\
                            = & ~ \sum_{j = 0}^p \Bigg( \sum_{i = 0}^p 
                                  Q_i Q_j(x) \, \A^i \Bigg) \tA^{\,j} \\
                            = & ~ \sum_{j = 0}^p \Bigg( \sum_{i = 0}^j w_0^i \, 
                                  Q_i Q_j(x) \, \tA^{\,j - i} 
                                + \sum_{i = j + 1}^p w_0^{\,j} \, Q_i Q_j(x) \, 
                                \A^{i - j} \Bigg) \\
                            = & ~ \sum_{k = 0}^p \A^k \Bigg( \sum_{j = 0}^p 
                                  \sum_{i = 0}^j w_0^i \, d_{k,\,j - i} \, 
                                  Q_i Q_j(x) + \sum_{j = 0}^{p - k} w_0^{\,j} \, 
                                  Q_{k + j} Q_j (x) \Bigg) 
 \end{split}
\]
where $d_{k,0} = 0$ if $k > 0$.  Write the expression in the last line above as 
$\sum_{k = 0}^p \Psi_k(x) \, \A^k$.  For $1 \leq k \leq p$, the vanishing of 
each $\Psi_k$ then gives the stated relation for $Q_k Q_0$.  



\subsection{The commutation relations}
\label{subsec:comm}

To facilitate computations, we perform a change of variables 
\[
 \B \ce \A + (-1)^{\,p} 
\]
We then have 
\[
 \begin{split}
  \psi^p(h x) = & ~ \psi^p(h) \, \psi^p(x) \hskip 7.7cm \\
              = & ~ \sum_{i = 0}^p Q_i(h) \, \A^i \, \sum_{j = 0}^p Q_j(x) \, 
                    \A^{\,j} \\
              = & ~ \sum_{m = 0}^{2 p} \Bigg(\!\! 
                    \sum_{\stackrel{\scriptstyle i + j = m}
                    {0 \,\leq\, i, \, j \,\leq\, p}} Q_i(h) \, Q_j(x) \Bigg) 
                    \A^m 
 \end{split}
\]
\[
 \begin{split}
              = & ~ \sum_{m = 0}^{2 p} \Bigg(\!\! 
                    \sum_{\stackrel{\scriptstyle i + j = m}
                    {0 \,\leq\, i, \, j \,\leq\, p}} Q_i(h) \, Q_j(x) \Bigg) 
                    \big(\B + (-1)^{\,p + 1}\big)^m \\
              = & ~ \sum_{m = 0}^{2 p} \Bigg(\!\! 
                    \sum_{\stackrel{\scriptstyle i + j = m}
                    {0 \,\leq\, i, \, j \,\leq\, p}} Q_i(h) \, Q_j(x) \Bigg) 
                    \sum_{n = 0}^m \ch{m}{n} \B^n (-1)^{(p + 1) (m - n)} \\
              = & ~ \sum_{n = 0}^{2 p} e_n \, \B^n 
 \end{split}
\]
where 
\[
  e_n = \sum_{m = n}^{2 p} (-1)^{(p + 1) (m - n)} \ch{m}{n} \! 
        \sum_{\stackrel{\scriptstyle i + j = m}{0 \,\leq\, i, \, j \,\leq\, p}} 
        Q_i(h) \, Q_j(x) 
\]
Formulas for the terms $Q_i(h)$ above are given by Theorem \ref{thm:po}\,(ii).  
Note that $Q_1(h)$ includes a term of 1.  

We now reduce $\psi^p(h x)$ above modulo $w(h,\A)$, by first rewriting the 
latter as a polynomial in $\B$: 
\[
 \begin{split}
  w(h,\A) = & ~ (\A - p) \big(\A + (-1)^{\,p}\big)^p 
              - \big(h - p^2 + (-1)^{\,p}\big) \A \\
          = & ~ \big(\B + (-1)^{\,p + 1} - p\big) \B^{\,p} 
              - \big(h - p^2 + (-1)^{\,p}\big) \big(\B + (-1)^{\,p + 1}\big) \\
          = & ~ \B^{\,p + 1} + v_p \, \B^{\,p} + v_1 \, \B + v_0 
 \end{split}
\]
where 
\[
 v_p = (-1)^{\,p + 1} - p, \quad v_1 = -\big(h - p^2 + (-1)^{\,p}\big), 
 \quad \ad \quad v_0 = (-1)^{\,p + 1} v_1 
\]
Performing long division of $\psi^p(h x)$ by $w(h,\A)$ with respect to $\B$, we 
get 
\[
 \psi^p(h x) \equiv \sum_{j = 0}^p \, f_{\,j} \, \B^{\,j} \md w(h,\A) 
\]
where 
\[
 f_{\,j} = \left\{ \hskip -.1cm
  \begin{array}{lll}
   \displaystyle e_p - v_1 e_{2 p} + v_p \sum_{m = 0}^{p - 1} (-1)^{m + 1} 
                 e_{p + 1 + m} \, v_p^m              && \,j = p \\
   \displaystyle e_j + v_0 \sum_{m = 0}^{p - j - 1} (-1)^{m + 1} 
                 e_{p + j + 1 + m} \, v_p^m + v_1 \sum_{m = 0}^{p - j} 
                 (-1)^{m + 1} e_{p + j + m} \, v_p^m && 0 < j < p \\
   \displaystyle e_0 + v_0 \sum_{m = 0}^{p - 1} (-1)^{m + 1} e_{p + 1 + m} \, 
                 v_p^m                               && \,j = 0 
  \end{array}
 \right. 
\]
Thus we can rewrite 
\[
 \begin{split}
  \psi^p(h x) = & ~ \sum_{j = 0}^p \, f_{\,j} \, \big(\A + (-1)^{\,p}\big)^j \\
              = & ~ \sum_{j = 0}^p \, f_{\,j} \, \sum_{i = 0}^j \ch{\,j\,}{i} 
                    \A^i \, (-1)^{\,p \, (\,j - i)} \\
              = & ~ \sum_{i = 0}^p \Bigg[ \sum_{j = i}^p (-1)^{\,p \, (\,j - i)} 
                    \ch{\,j\,}{i} \, f_{\,j} \Bigg] \A^i 
 \end{split}
\]

On the other hand, $\psi^p(h x) = \sum_{i = 0}^p Q_i (h x) \, \A^i$.  Comparing 
this to the last expression for $\psi^p(h x)$ above, term by term, we obtain the 
commutation relations as stated.  \qed



% \bibliographystyle{amsalpha}
% \bibliography{me}
% \end{document}

\renewcommand\refname{}
\newcommand{\AX}[1]{\href{http://arxiv.org/abs/#1}{arXiv:#1}}
\newcommand{\MRn}[2]{\href{http://www.ams.org/mathscinet-getitem?mr=#1}{MR#1#2}}
\newcommand{\name}{TateNormalLevelResolutions.pdf}
\wt{.}\vspace{-.5in}
\begin{thebibliography}

\section*{\leftskip=-.44in References \vspace{.17in}}

\bibitem[Ahlgren2003]{Ahlgren}
Scott Ahlgren, \emph{The theta-operator and the divisors of modular forms on
  genus zero subgroups}, Math. Res. Lett. \textbf{10} (2003), no.~6,
  787--798. \MRn{2024734}{(2004m:11059)}

\bibitem[Ando1995]{Ando95}
Matthew Ando, \emph{Isogenies of formal group laws and power operations in the
  cohomology theories {$E\sb n$}}, Duke Math. J. \textbf{79} (1995), no.~2,
  423--485. \MRn{1344767}{(97a:55006)}

\bibitem[Ando-Hopkins-Strickland2004]{AHS04}
Matthew Ando, Michael~J. Hopkins, and Neil~P. Strickland, \emph{The sigma
  orientation is an {$H\sb \infty$} map}, Amer. J. Math. \textbf{126} (2004),
  no.~2, 247--334. \MRn{2045503}{(2005d:55009)}

\bibitem[Bruinier-Kohnen-Ono2004]{BKO}
Jan~H. Bruinier, Winfried Kohnen, and Ken Ono, \emph{The arithmetic of the
  values of modular functions and the divisors of modular forms}, Compos. Math.
  \textbf{140} (2004), no.~3, 552--566. \MRn{2041768}{(2005h:11083)}

\bibitem[Choi2006]{Choi}
D.~Choi, \emph{On values of a modular form on {$\Gamma\sb 0(N)$}}, Acta Arith.
  \textbf{121} (2006), no.~4, 299--311. \MRn{2224397}{(2006m:11051)}

\bibitem[Cox2013]{Cox}
David~A. Cox, \emph{Primes of the form {$x\sp 2 + ny\sp 2$}}, second ed., Pure
  and Applied Mathematics (Hoboken), John Wiley \& Sons, Inc., Hoboken, NJ,
  2013, Fermat, class field theory, and complex multiplication. \MRn{3236783}{}

\bibitem[Katz1973]{padicprop}
Nicholas~M. Katz, \emph{{$p$}-adic properties of modular schemes and modular
  forms}, Modular functions of one variable, {III} ({P}roc. {I}nternat.
  {S}ummer {S}chool, {U}niv. {A}ntwerp, {A}ntwerp, 1972), Springer, Berlin,
  1973, pp.~69--190. Lecture Notes in Mathematics, Vol. 350. \MRn{0447119}{(56
  \#5434)}

\bibitem[Katz-Mazur1985]{KM}
Nicholas~M. Katz and Barry Mazur, \emph{Arithmetic moduli of elliptic curves},
  Annals of Mathematics Studies, vol. 108, Princeton University Press,
  Princeton, NJ, 1985. \MRn{772569}{(86i:11024)}

\bibitem[Lubin1967]{Lubin}
Jonathan Lubin, \emph{Finite subgroups and isogenies of one-parameter formal
  {L}ie groups}, Ann. of Math. (2) \textbf{85} (1967), 296--302. 
  \MRn{0209287}{(35 \#189)}

\bibitem[Lubin-Serre-Tate1964]{LST}
J.~Lubin, J.-P. Serre, and J.~Tate, \emph{Elliptic curves and formal groups},
  available at \,\! \href{http://www.ma.utexas.edu/users/voloch/lst.html}
  {http://www.ma.utexas.edu/users/voloch/lst.html}.

\bibitem[Lubin-Tate1966]{LubinTate}
Jonathan Lubin and John Tate, \emph{Formal moduli for one-parameter formal
  {L}ie groups}, Bull. Soc. Math. France \textbf{94} (1966), 49--59.
  \MRn{0238854}{(39 \#214)}

\bibitem[Milne2012]{Milne}
J.S. Milne, \emph{Modular functions and modular forms}, available at \\
  \href{http://www.jmilne.org/math/CourseNotes/MF.pdf}
  {http://www.jmilne.org/math/}.

\bibitem[Quillen1969]{Quillen}
Daniel Quillen, \emph{On the formal group laws of unoriented and complex
  cobordism theory}, Bull. Amer. Math. Soc. \textbf{75} (1969), 1293--1298.
  \MRn{0253350}{(40 \#6565)}

\bibitem[Rezk2008]{h2p2}
Charles Rezk, \emph{Power operations for {M}orava {$E$}-theory of height 2 at 
  the prime 2}. \AX{0812.1320}

\bibitem[Rezk2009]{cong}
Charles Rezk, \emph{The congruence criterion for power operations in {M}orava
  {$E$}-theory}, Homology, Homotopy Appl. \textbf{11} (2009), no.~2, 327--379.
  \MRn{2591924}{(2011e:55021)}

\bibitem[Rezk2012]{mc1}
Charles Rezk, \emph{Modular isogeny complexes}, Algebr. Geom. Topol. \textbf{12}
  (2012), no.~3, 1373--1403. \MRn{2966690}{}

\bibitem[Strickland1997]{Str97}
Neil~P. Strickland, \emph{Finite subgroups of formal groups}, J. Pure Appl.
  Algebra \textbf{121} (1997), no.~2, 161--208. \MRn{1473889}{(98k:14065)}

\bibitem[Strickland1998]{Str98}
N.~P. Strickland, \emph{Morava {$E$}-theory of symmetric groups}, Topology
  \textbf{37} (1998), no.~4, 757--779. \MRn{1607736}{(99e:55008)}

\bibitem[Zhu2014]{p3}
Yifei Zhu, \emph{The power operation structure on {M}orava {$E$}-theory of
  height 2 at the prime 3}, Algebr. Geom. Topol. \textbf{14} (2014), no.~2,
  953--977. \MRn{3160608}{}

\bibitem[Zhu2015]{ho}
Yifei Zhu, \emph{The {H}ecke algebra action on {M}orava {$E$}-theory of height
  2}. \AX{1505.06377}

\end{thebibliography}



\end{document}